\begin{abstract}
\addchaptertocentry{\abstractname}

This dissertation aims to improve the accessibility and interpretability of biological data stored in the GeneNetwork (GN) repository for machine analysis.  The GN database contains over 20 years of experimental data from genetics, phenotyping, QTL, and GWA studies in human and model species, such as mouse and rat.  However, the data is currently difficult to access and manipulate due to its complex underlying structures, including around 80 cross-referenced SQL tables and various file types.  To address these challenges, this dissertation proposes to reorganize the data and provide automated data discovery using graph database technology.  This will enable the creation of a new, more flexible GN service and the ability to automatically infer relationships between data elements.  Overall, the expected outcome of this dissertation is to enhance the value of the GN service through improved data retrieval and to allow for complex query capabilities and to map out future storage and retrieval techniques that are relevant for AI research on biomedical data.

\textbf{\textsc{Keywords:}} \textit{\keywordnames}
\end{abstract}
