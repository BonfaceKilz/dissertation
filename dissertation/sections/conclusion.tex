\chapter{Conclusion}

In conclusion, this dissertation has explored the practical implementation and evaluation of a self-documenting DSL designed to transform SQL tables into RDF, with a specific focus on addressing limitations within the GeneNetwork platform related to SQL queries. Through the development and testing of this DSL, several key findings and outcomes have emerged.

Firstly, the performance analysis conducted between SQL and SPARQL queries has demonstrated significant advantages of using RDF over SQL, particularly in terms of query execution times. Across multiple benchmarks, SPARQL consistently exhibited faster response times compared to its SQL counterparts, especially as query complexity increased. This performance advantage is critical for enhancing the overall responsiveness of production systems like GeneNetwork.

Furthermore, beyond the performance benefits, RDF offers a more intuitive and straightforward approach to query formation. By leveraging RDF's graph-based structure and semantics, query construction becomes more transparent and comprehensible compared to SQL, which often relies on complex join operations.

Lastly, RDF's ability to query external web services and integrate data from diverse sources presents a compelling advantage for data exploration and integration. By enabling federated queries and seamless access to external datasets, RDF extends the capabilities of systems like GeneNetwork to incorporate a broader range of data resources, ultimately empowering researchers and developers to derive deeper insights from distributed and heterogeneous data.

In summary, the findings from this dissertation underscore the importance and advantages of leveraging RDF-based solutions, such as the developed DSL, to address complex data integration and querying challenges within bioinformatics and related domains. The demonstrated performance improvements and enhanced query flexibility highlight the potential for RDF to contribute significantly to advancing data-driven research and applications within genomic and phenotypic data analysis.
