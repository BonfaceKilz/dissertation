\chapter{Conclusions, Recommendations and Future Work}

\section{Conclusion}

This dissertation has explored the practical implementation and evaluation of a self-documenting DSL designed to transform SQL tables into RDF, with a specific focus on addressing limitations within the GeneNetwork platform related to SQL queries. Through the development and testing of this DSL, several key findings and outcomes have emerged.

Firstly, the performance analysis conducted between SQL and SPARQL queries has demonstrated significant advantages of using RDF over SQL, particularly in terms of query execution times. Across multiple benchmarks, SPARQL consistently exhibited faster response times compared to its SQL counterparts, especially as query complexity increased. This performance advantage is critical for enhancing the overall responsiveness of production systems like GeneNetwork.

Furthermore, beyond the performance benefits, RDF makes data FAIR.  By making data FAIR, RDF extends the capabilities of systems like GeneNetwork to incorporate a broader range of data resources, ultimately empowering researchers and developers to derive deeper insights from distributed and heterogeneous data.

\section{Recommendations}

Researchers and developers within the domain of bioinformatics and data management can leverage the developed DSL to streamline the transformation of SQL data into RDF, enhancing data accessibility and interoperability. Institutions and organizations involved in data-intensive research can benefit from adopting RDF for improved data management, search capabilities, and adherence to FAIR data principles.

The generated RDF can be used to represent data as a Knowledge Graph over an extensible ontology.  This knowledge graph can be used as the basis of Retrieval-Augmented Language Models that use some input sequence to search the knowledge graph and fuse the results with the language model output to generate more useful responses.

\section{Future Work}

Beyond creating a DSL that can convert SQL into RDF, a more flexible reasoning system based off a logic programming language such as Prolog, should be created to read the generated RDF.  Such a system could offer enhanced capabilities for semantic reasoning and inference, leveraging the Logic programming language's logic-based approach to derive insights and conclusions from RDF data. Future work in this direction could involve exploring the integration of this reasoning system with advanced AI techniques, such as deep learning and natural language processing, to enable more sophisticated and context-aware reasoning capabilities. Additionally, efforts could focus on developing novel algorithms and inference mechanisms tailored to the specific challenges and requirements of RDF data, ultimately leading to more intelligent and adaptive systems for knowledge representation and semantic analysis.

%%% Local Variables:
%%% ispell-local-dictionary: "en_GB-ise"
%%% End:
