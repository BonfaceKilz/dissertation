\chapter{Introduction}

\section{Background}
Genetic data analysis is important in understanding the underlying mechanisms of various biological processes and diseases.  Genenetwork2 is a powerful open-source software platform that provides a range of tools and resources for analyzing, visualizing, and storing genetic data \citep{sloan2016genenetwork,mulligan2017genenetwork}.  It contains over 20 years of experimental data from genetics, phenotyping, QTL, and GWA studies in human and model species such as mouse and rat \citep{sloan2016genenetwork}.  However, the current SQL database underlying GN can be complex and difficult to use, and lacks interoperability with other systems.

The Resource Description Framework (RDF) is a widely-used semantic data model that can provide several benefits over SQL, including support for complex data types, reasoning, and ontologies \citep{candan2001resource}.  Representing GN's data in RDF has the potential to enable more flexible querying, data integration, and reasoning capabilities, as well as better support for Linked Data and Semantic Web standards.

\section{Problem Statement}

The current SQL database of Genenetwork2 has several limitations, such as its lack of support for complex data types, reasoning, and ontologies.  These limitations make it difficult to integrate and query Genenetwork2's data with other sources, and to perform advanced analysis tasks, such as inferencing and querying by example.

\section{General Objectives}

The main objective of this dissertation is to port Genenetwork2's SQL database to RDF, and to evaluate the potential benefits and challenges of this approach.  The specific research objectives are:

\begin{enumerate}
\item Evaluate the potential benefits and challenges of porting the database to RDF, including its impact on data interoperability, integration, and security; and
\item Develop a detailed plan for the porting process, including the steps, tools, and resources.
\end{enumerate}


\section{Research Objectives and Research Questions}

\subsection*{Research Objectives:}

\begin{enumerate}
\item Understand the current limitations of the Genenetwork2 SQL database and how the use of RDF could address these limitations, as well as the potential benefits and challenges of porting the Genenetwork2 database to RDF, including its impact on data interoperability, integration, and security.
  \item To examine previous research and studies conducted on the GeneNetwork platform.
\item Design and test a framework for the porting process, including steps, tools, and resources, for porting the Genenetwork2 database to RDF.
\item Validate the research by evaluating the benefits and challenges of the ported Genenetwork2 database in RDF, including its impact on data interoperability, integration, and security.
\end{enumerate}

\subsection*{Research Questions:}

\begin{enumerate}
\item What are the current limitations of the Genenetwork2 SQL database and how could the use of RDF address these limitations?
\item How can the developed plan for porting the Genenetwork2 database to RDF be tested?
\end{enumerate}
   

\section{Scope and Limitations}
\subsection*{Scope:}

The scope of this dissertation is to port the Genenetwork2 SQL database to RDF and evaluate the potential benefits and challenges of this approach.  This dissertation will focus on the potential impact of the porting on data interoperability, integration, and security.  The research will also involve developing a detailed plan for the porting process, including steps, tools, and resources.  Access to one of the University of Tennessee Health Science Center's (UTHSC) servers, which enables the use of a large database at scale, will facilitate this process.

\subsection*{Limitations:}

\begin{enumerate}
\item This dissertation will only focus on the porting of the Genenetwork2 SQL database to RDF and will not consider other potential solutions or approaches.
\item This dissertation will not address any other issues or limitations related to Genenetwork2 beyond the current SQL database.
\end{enumerate}
