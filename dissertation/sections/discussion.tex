\chapter{Discussion of Results}

\section{Introduction}

This dissertation aimed to investigate prior research and studies concerning the GeneNetwork platform and RDF to identify the existing limitations of GeneNetwork related to SQL and explore how RDF could potentially mitigate these limitations.  Additionally, a domain-specific language (DSL) was developed and implemented to transform GeneNetwork metadata from SQL to RDF, followed by performance bench marking and output quality assessments.  This chapter discusses the performance benchmarks and output quality assessment that were conducted in addition to highlighting additional benefits that can be gained by using RDF to store metadata.

\section{Performance Analysis}

The conducted performance tests compared SQL and SPARQL queries for retrieving information related to diabetes from a GeneNetwork SQL store.  The initial SQL query aimed to count the number of publications with abstracts containing the term ``diabetes''.  Correspondingly, the SPARQL query transformed this SQL query into RDF format to achieve the same result using RDF data model constructs.  More complicated queries were constructed to benchmark the performance of RDF against SQL.\@

Table~\ref{table:perf-1}, Table~\ref{table:perf-2} and Table~\ref{table:perf-3} show the comparison in preformance between SQL queries against MariaDB and SPARQL against Virtuoso.  These performance results highlight notable disparities in execution times between SQL and SPARQL queries.  Across multiple executions, the SPARQL query consistently exhibited faster response times compared to its SQL equivalent.  For example, the SQL query executed within milliseconds (0.08~-~0.11 seconds), whereas the equivalent SPARQL query consistently completed in mere fractions of a second (0.013~-~0.017 seconds).  Furthermore, as the complexity of the queries increased, SQL experienced a more pronounced slowdown compared to SPARQL\@.

\section{Output Quality and FAIR Data}

FAIR data means that data is Findable, Accessible, Interoperable and Re-usable.  Figure~\ref{sql:mouse-results} shows the results of executing a SQL query shown in Figure~\ref{sql:mouse}.  This data is not annotated.  In SQL, to create extra annotations, one would have to create extra columns or create new tables to store this extra metadata.  However, in RDF, the result set is richer as shown in the json-ld formats demonstrated in the previous chapter.  This data is FAIR because:

\begin{enumerate}
\item \textbf{Findable}: The data includes unique identifiers (URIs) for each entity, such as the species "Glossophaga\_soricina" and its properties, making it easily locatable on the web. Additionally, it adheres to common standards such as SKOS (Simple Knowledge Organization System) for labelling and categorisation, enhancing its discoverability.
\item \textbf{Accessible}: The data is publicly available and accessible through the provided URIs. This ensures that researchers and interested parties can retrieve the information without restrictions, promoting transparency and openness.
\item \textbf{Interoperable}: The data is represented in RDF (Resource Description Framework), a widely used format for expressing linked data. This facilitates integration with other datasets and systems, enabling seamless interoperability across different platforms and applications.
\item \textbf{Reusable}: The data is structured and annotated with descriptive properties, enhancing its usability for various purposes. By adhering to established vocabularies and standards, such as SKOS and RDF, the data becomes more interpretable and reusable in different contexts, contributing to its overall re-usability.
\end{enumerate}


\section{Advantages of RDF for Complex Queries}

In lieu of the preceding performance analysis and output assessment, the significance of leveraging RDF instead of SQL to store complex relationships becomes clear:

Firstly, RDF shows query execution times compared to SQL.\@  In SQL, the performance of queries tends to degrade with increasing complexity, often due to the need for multiple joins.  However, RDF queries do not exhibit this performance decline associated with joins in SQL.\@  Comparative evaluations of equivalent queries in SQL and RDF consistently show that RDF outperforms SQL for complex queries, moreso ones with a tree-like structure.  This performance advantage is particularly crucial in production systems where responsiveness to end users is critical.  In the context of Genenetwork, leveraging fast query execution provided by RDF can enhance its search capabilities significantly.

In addition to the evident performance advantages, adopting RDF for data modelling confers a significant contribution to the principles of FAIR data.  By structuring data in RDF, information becomes inherently more findable, as each entity is assigned a unique identifier, facilitating easy discovery through standardised querying methods.  Moreover, RDF-encoded data enhances accessibility by providing open and transparent access points via web protocols, ensuring that data can be retrieved without constraints or barriers.  Interoperability is also greatly promoted through RDF modelling, as it enables seamless integration, through federated queries, with other datasets and systems, fostering data exchange and interoperable interactions across heterogeneous environments.  Additionally, RDF's semantic nature and adherence to standardised vocabularies enhance data reusability, as it ensures clarity and consistency in data interpretation, enabling its effective utilisation across diverse applications and domains.

%%% Local Variables:
%%% ispell-local-dictionary: "en_GB-ise"
%%% End:
