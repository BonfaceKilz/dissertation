\section{Methodology}

The objective of this dissertation is to enhance the accessibility and interpretability of biological metadata in GN by using RDF which will be utilized to reorganize the data and facilitate automated data discovery, with the ultimate goal of improving metadata retrieval and complex query capabilities.  This chapter provides an overview of the tools and data that will be used in this dissertation, where the data is sourced from, the design and implementation of the SQL to RDF dumper, including testing and validation details, and the expected results of this dissertation and how they will be presented.

\subsection{Tools and Software Used}

For this dissertation, the following tools and software will be used:

\begin{enumerate}
\item \textit{GNU/Linux}:  MacOS and Windows are not factored in at all in the development environment.  A GNU/Linux distribution is used because it is free, open and in wide usage for software development and deployment purposes.  MacOS and Windows are expensive.
\item \textit{GNU Guix}: This will act as a package manager to declaratively specify and control the dependencies for the dissertation, and guarantee bit for bit reproducibility.
\item \textit{GNU Guile}: GNU Guile, as a dialect of Scheme, will be the primary programming language utilized to develop the SQL to RDF parser. Its robust macro system will facilitate the creation of a declarative and expressive Domain Specific Language that will greatly aid in parsing SQL to RDF.
\item \textit{MariaDB}: This is an open-source RDBMS that is a fork of MySQL which can be used as a drop-in replacement for MySQL.  MariaDB will primarily be used as the SQL data source in this dissertation.
\item \textit{Virtuoso}: This is an open source high-performance engine that can be use to manage and query large amounts of RDF data.
\item \textit{Graphviz}: This is an open-source graph visualization software that is used to visualize different types of graphs.
\item \textit{rapper}: This is a RDF parsing and serializing utility that can be used to lint RDF files that use the Turtle syntax.
\item \textit{pytorch or GPT-3}: We will use a machine learning tool to scope AI needs. pytorch is a good candidate.  We will also try natural language processing, such as offered by GPT-3 and feed RDF to see what type of questions we can ask and provide future recommendations.
\end{enumerate}

\subsection{Data Collection}

The GN small database, a publicly accessible and reduced version of the main database, will be used as the primary source of data during the implementation of this dissertation.  The main database, which is constantly growing, hosted at the University of Tennessee Health Science Center, will be used to test and validate the results.  Testing and validation of the results will be done against the main database which is hosted at the University of Tennessee Health Science Center.

In addition to the GeneNetwork2 database, other relevant datasets from external sources may also be collected and included in the analysis.  These sources may include publicly available biological databases, such as the Gene Ontology database or the Uniprot protein database.  Careful consideration will be given to the selection of these external datasets to ensure that they are relevant and reliable.

Overall, the data collection process for this research project will involve extracting the data from the GeneNetwork2 database and potentially including additional relevant datasets from external sources.

\subsection{Design and Implementation}

GN currently operates using SQL on a MariaDB instance.  The first step is to create a parser that can be used to convert the results of a SQL query to a valid triple store.

The initial step in creating the SQL to RDF parser is to establish a set of macros that enable us to dump data in a declarative manner.  For this dissertation, a \texttt{define-dump} macro will be used to dump a view of the database.  It will consist of three order-agnostic clauses - tables, schema-triples and triples - in the form show below:

\begin{verbatim}
(define-dump function-name
  (tables (table ...) raw-forms ...)
  (schema-triples
   (subject predicate object) ...)
  (triples subject
    (verb predicate object) ...))
\end{verbatim}

The \texttt{tables} clause specifies the database tables to be joined to construct the view to be dumped.  \texttt{TABLE} must be either of the form \texttt{TABLE-NAME} or of the form \texttt{JOIN-OPERATOR TABLE-NAME RAW-CONDITION}.  \texttt{TABLE-NAME} is the name of the target table.  The \texttt{JOIN-OPERATOR} can be either a join, left-join or an inner-join.  \texttt{RAW-CONDITION} should be the join condition as a raw string for example \texttt{``USING (SpeciesId)''}.  \texttt{RAW-FORMS} are expressions that should evaluate to strings which will be appended to the SQL query.

The schema-triples clause specifies the list of triples to be written once when the dump starts.

The triples clause specifies the triples to be dumped once for each row in the view. All triples have a common \texttt{SUBJECT}. The \texttt{(verb predicate object)} clauses are described below:

\begin{itemize}
\item \texttt{VERB} must either be a set or multiset. For the set \texttt{VERB}, a single triple \texttt{(SUBJECT PREDICATE OBJECT-VALUE)} is written where \texttt{OBJECT-VALUE} is the result of evaluating \texttt{OBJECT}. For the multiset \texttt{VERB}, \texttt{OBJECT} must evaluate to a list, and a triple \texttt{(SUBJECT PREDICATE OBJECT-VALUE-ELEMENT)} is created for each element \texttt{OBJECT-VALUE-ELEMENT} of that list.
\item The \texttt{SUBJECT} and \texttt{OBJECT} expressions in the triples clause must reference database fields using a \texttt{(field TABLE COLUMN)} clause where \texttt{TABLE} and \texttt{COLUMN} refer to the table and column of the field being referenced. Database fields can also be referenced using \texttt{(field TABLE COLUMN ALIAS)} where \texttt{ALIAS} is an alias for that column in the SQL query.
\end{itemize}

As an example, consider the following ``Species'' table which is defined as follows:

\begin{center}
\begin{tabular}{llllll}
Field & Type & Null & Key & Default & Extra\\
\hline
Id & smallint(5) unsigned & NO & PRI & NULL & auto\_increment\\
SpeciesId & int(5) & YES &  & NULL & \\
SpeciesName & varchar(50) & YES &  & NULL & \\
Name & char(30) & NO & MUL &  & \\
MenuName & char(50) & YES &  & NULL & \\
FullName & char(100) & NO &  &  & \\
TaxonomyId & int(11) & YES &  & NULL & \\
OrderId & smallint(6) & YES &  & NULL & \\
\hline
\end{tabular}
\end{center}

In the small GN database, the result of executing \textit{``SELECT * FROM Species;''} is:

\begin{center}
\begin{tabular}{rrllllrr}
Id & SpeciesId & SpeciesName & Name & MenuName & FullName & TaxonomyId & OrderId\\
\hline
1 & 1 & Mouse & mouse & Mouse & Mus musculus & 10090 & 30\\
4 & 4 & Human & human & Human & Homo sapiens & 9606 & 10\\
\hline
\end{tabular}
\end{center}


As an example, using the ``Species'' table outlined above, the dump in GNU Guile would be defined in a declarative form as follows:

\begin{verbatim}
(define-dump dump-species
  (tables (Species))
  (schema-triples
   (gn:name rdfs:range rdfs:Literal)
   (gn:menuname rdfs:range rdfs:Literal)
   (gn:binomialName rdfs:range rdfs:Literal))
  (triples (binomial-name->species-id (field Species FullName))
    (set gn:name (field Species SpeciesName))
    (set gn:menuName (field Species MenuName))
    (set gn:binomialName (field Species FullName))))
\end{verbatim}

The above snippet defines a dump called ``dump-species'' which is used to dump data from the ``Species'' table.  It specifies the schema triples for the dump, which includes the properties: \textit{gn:name} \textit{gn:menuname} \textit{binomialName}.  To run the above macro, one would have to call it with a proper MariaDB connection as such:

\begin{verbatim}
(call-with-genenetwork-database
 (lambda (db)
   (with-output-to-file (string-append %dump-directory "/dump.ttl")
     (lambda ()
       (prefix "rdf:" "<http://www.w3.org/1999/02/22-rdf-syntax-ns#>")
       (prefix "rdfs:" "<http://www.w3.org/2000/01/rdf-schema#>")
       (prefix "foaf:" "<http://xmlns.com/foaf/0.1/>")
       (prefix "gn:" "<http://genenetwork.org/>")
       (newline)
       (dump-species db)))))
\end{verbatim}

The above would result in the following RDF:

\begin{verbatim}
@prefix rdf: <http://www.w3.org/1999/02/22-rdf-syntax-ns#> .
@prefix rdfs: <http://www.w3.org/2000/01/rdf-schema#> .
@prefix foaf: <http://xmlns.com/foaf/0.1/> .
@prefix gn: <http://genenetwork.org/> .

gn:name rdfs:range rdfs:Literal .
gn:menuname rdfs:range rdfs:Literal .
gn:binomialName rdfs:range rdfs:Literal .
gn:dump_species_name rdf:type gn:dump .
gn:dump_species_name gn:createsPredicate gn:name .
gn:dump_species_name gn:dependsOn gn:field_species__speciesname .
gn:dump_species_menuname rdf:type gn:dump .
gn:dump_species_menuname gn:createsPredicate gn:menuName .
gn:dump_species_menuname gn:dependsOn gn:field_species__menuname .
gn:dump_species_binomialname rdf:type gn:dump .
gn:dump_species_binomialname gn:createsPredicate gn:binomialName .
gn:dump_species_binomialname gn:dependsOn gn:field_species__fullname .
gn:species_mus_musculus gn:name "Mouse" .
gn:species_mus_musculus gn:menuName "Mouse" .
gn:species_mus_musculus gn:binomialName "Mus musculus" .
gn:species_homo_sapiens gn:name "Human" .
gn:species_homo_sapiens gn:menuName "Human" .
gn:species_homo_sapiens gn:binomialName "Homo sapiens" .
\end{verbatim}


\subsection{Validation and Testing}
\subsubsection*{Validation}
In order to ensure that the output of the SQL to RDF parser is valid, \textit{rapper}, an RDF linter, will be used to assess whether there are any errors in the output of the RDF format.  This will make sure that there are no errors thereby allowing data to be ingested into RDF.  Should there be no errors, the data will be uploaded to a running Virtuoso instance for testing.

\subsubsection*{Testing}

In order to evaluate the performance and efficiency of RDF as a data representation and querying format, a subset of the functionality provided by Genenetwork2, which is currently based on SQL, will be replaced with RDF.  This will enable a direct comparison of the speed and complexity of RDF vs SQL.  By measuring the time and resources required to perform specific tasks, we will be able to gain a deeper understanding of the trade-offs involved in using RDF, and how it compares to SQL in terms of performance, complexity and scalability.


\subsection{Expected Results}
The expected results are:

\begin{enumerate}
\item A thorough analysis of Genenetwork2's current SQL database.

\item A detailed evaluation of the potential benefits and challenges of porting the database to RDF, including its impact on data interoperability, integration, and security.

\item A proof-of-concept implementation of the RDF-based Genenetwork2 database, including the necessary scripts and tools for data migration and integration.

\item An assessment of the performance and interoperability of the RDF-based database.

\item An evaluation of the potential benefits of the RDF-based Genenetwork2 for users and the broader genetics community, including its impact on data integration, analysis, and sharing.
\end{enumerate}
